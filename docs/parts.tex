\section{Parts}

\begin{tabular}{| l | c | l | l |}
\hline
Part & Icon & Base Price & Base Tech \\
\hline
Control Node & \includegraphics[scale=0.70]{images/part-controlnode.png} & & \\
Flux Modulator & \includegraphics[scale=0.70]{images/part-fluxmodulator.png} & &\\
Photon Bridge & \includegraphics[scale=0.70]{images/part-photonbridge.png} & &\\
ST Distorter & \includegraphics[scale=0.70]{images/part-stdistorter.png} & &\\
Cryofuse & \includegraphics[scale=0.70]{images/part-cryofuse.png} & &\\
Plasma Tube & \includegraphics[scale=0.70]{images/part-plasmatube.png} & &\\
Scanner & \includegraphics[scale=0.70]{images/part-scanner.png} & &\\
J Junc Module & \includegraphics[scale=0.70]{images/part-jjunc.png} & &\\
\hline
\end{tabular}

\subsubsection{Control Node}
Control Nodes are data processing controllers for each system rail, and as such 
they can not be shunted. Thus it's necessary to have a reasonable amount of 
them on hand to keep the ship running smoothly.  

\subsubsection{Flux Modulator}
Flux Modulators are used with the shields, sub-c engines and warp drives.

\subsubsection{Photon Bridge}
Photon bridges are used with tactical and the warp engines.

\subsubsection{ST Distorter}
Space/time distorters (ST Distorters) are used only with the warp and
sub-light (sub-c) engines. This allows the ship to form a bubble of space-time around itself and propel this bubble faster than light-speed, thus enabling the bending of the laws of General Relativity.

\subsubsection{Cryofuse}

Cryofuses are designed to maintain circuit continuity and serve as a fuse to avoid 
damage to the system. Two components with different critical temperatures are suspended 
in a tube of liquid helium within the cryofuse. The magnetic field between the two 
components can be instantaneously adjusted to meet the superconductive needs of the 
system, up to and including temporary interruption of data flow when the fuse trips.  

\subsubsection{Plasma Tube}

Plasma tubes are used as part of the cannon and laser systems aboard the
SunDog. This allows the focus and momentary retention of ship's power prior to firing a 
weapons-grade bolt of energy without damaging the ship's subsystems in the process. 

\subsubsection{Scanner}

Scanners, located in the Pilotage and Tactical bays, enable the ship to map the ground terrain, search for star ports, and to sense the space surrounding the ship. 

\subsubsection{J Junc Module}

This is a switching circuit used in the Pilotage and Tactical bays, and controls the flow of sensory requests and input. 

\subsubsection{Shunt}

A shunt is a temporary replacement part that can fill the spot used by any part other than the Control Node. A shunted rail position will only function at 50% of its normal capacity, thus lowering the system's health accordingly. 

\subsection{Special Parts}
A number special parts can be added to the Sun Dog to give it extra
capabilities.

\begin{tabular}{| l | c | l | l |}
\hline
Part & Icon & Base Price & Base Tech \\
\hline
Ground Scanner & \includegraphics[scale=0.70]{images/part-groundscanner.png} & & \\
Cloaker & \includegraphics[scale=0.70]{images/part-cloaker.png} & & \\
Decloaker & \includegraphics[scale=0.70]{images/part-decloaker.png} & & \\
Concentrator & \includegraphics[scale=0.70]{images/part-concentrator.png} & & \\
Booster & & 200,000 & Only found in Banville \\
\hline
\end{tabular}

Open Issues:
\begin{itemize}
\item Do we want to have an auto-slew like the Apple II version?
\item Is there an image for the booster?
\end{itemize}

\subsubsection{Ground Scanner}

The ground scanner gives the SunDog the capability to land at any city
on any planet, regardless of whether there is a space port located in
the city or not.  This capability works both from orbit, and for
city-to-city flights so that the ship does not have to leave the planet's
atmosphere (useful when wanting to avoid pirates).

\subsubsection{Cloaker}

The cloaker can be used to disguise the SunDog from its enemies.  This allows
the ship to evade cannon and laser fire from attacking ships until the
SunDog can land or warp to a different solar system.  Using the cloaker
drains a tremendous amount of fuel and will automatically shut off if there
is not enough fuel remaining to power it.

Open Issues:
\begin{itemize}
\item What is the rate of fuel drain for using the cloaker?
\end{itemize}

\subsubsection{Decloaker}

\subsubsection{Concentrator}

The concentrator allows the cannon and laser weapons to inflict more damage
than the standard system.

Open Issues:
\begin{itemize}
\item Is there a fuel penalty for using the concentrator?
\end{itemize}

\subsubsection{Booster}

The booster allows the SunDog to warp further distances than it normally
can travel.  This is the only way the ship can make it to Enlie in order
to pick up the remaining cryogens to complete Phase 9 of Banville (see the
Banville section).

Open Issues:
\begin{itemize}
\item When does the booster become available?
\end{itemize}


